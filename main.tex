%% \documentclass[a4paper]{article}
\documentclass[12pt]{article}

%% Language and font encoding
\usepackage[english]{babel}
\usepackage[utf8x]{inputenc}
\usepackage[T1]{fontenc}

\usepackage{tgtermes} % times font

%% Sets page size and margins
\usepackage[a4paper,top=3cm,bottom=2cm,left=3cm,right=3cm,marginparwidth=1.75cm]{geometry}

%% Useful packages
\usepackage{amsmath}
\usepackage{graphicx}
\usepackage[colorinlistoftodos]{todonotes}
\usepackage[colorlinks=true, allcolors=blue]{hyperref}
\usepackage[useregional]{datetime2}
\usepackage{array}
\usepackage{tabularx}
\usepackage{natbib}
\usepackage{authblk}
\usepackage{enumitem}
\usepackage{setspace}

\renewcommand{\baselinestretch}{1.25} 






\title{Evaluation of Community Detection Algorithms}
\author{Ping-Chang Lee}
\affil{MSc in Data Science \& Statistics \\ The University of Bath}
\date{{September}{2022}}

\begin{document}
\maketitle
\pagebreak

%% Access Page 1
    
\bigbreak\bigbreak\bigbreak
\bigbreak\bigbreak\bigbreak
\bigbreak\bigbreak\bigbreak

This dissertation may be made available for consultation within the University Library and may be photocopied or lent to other libraries for the purposes of consultation.
\bigbreak\bigbreak\bigbreak
Singed: ***My Signature Here***
\pagebreak


\tableofcontents
\pagebreak

\section{Introduction}
In recent decades, the study related to network analysis has drawn the attention of scholars form various fields. Among numerous topics of network analysis, discovering the community structure of network is one of the most essential approach to learn the complicated systems that can not easily be cracked by solely conducting shallow researches\cite{1}. 

\bigskip

Despite there are dozens of community detection algorithms already implemented to real-world network analysis and have already granted great success in many fields \cite{2,3}, the sample networks for performance testing are often simplified and sparse. In this dissertation, the LFR benchmark graph is introduced to mock the real-world network by increasing the complexity of the network. Afterwards, a subset of community detection algorithms are applied to the artificial network. Due to the nature of algorithms, the dissertation will introduce the computation complexity and their procedures. Also, since some algorithms are iterative, the dissertation selects the best result from iterations for evaluations under different circumstances. Next, the dissertation provides a series of approaches to evaluate the partitions of algorithms. As mentioned, a network can have either high complexity or very sparse structure, there is no consistent classification yet to determine if a given partition of a network is 'good'. The dissertation particularly underline the methodology on goodness of partition and the measure of similarity between the partition of a community detection algorithm and the ground truth community partition(network that is already labeled by communities).

\section{Background}
\subsection{Network Structure}

In general, there are two basic components that construct a network. A network(or a graph) G is composed by V and E, where V is the vertex(node) set containing all the vertices that belongs to the network and E is the edge(relation) set containing all the edge between vertices if two are connected. Further properties can be added to the network if the network is weighted or directed. In order to fit the goals of this dissertation, the label function L is add to the network, that is $G = \{V,E,L\}$. $L(v_i) = c_j$ indicates vertex $v_i$ belongs to community with label of $c_j$. With the definition of label stated previously, the community $C_i$ can be defined as $C_i = \{ L(v) = c_i \text{, } \forall v \in V \}$. 
Furthermore, given a partitioned network with n communities, let $C_i$ be the community with a label of $c_i$, then 
$\bigcup_{1 \leq i \leq n} C_i \subseteq V(G)$. When conducting analysis on a network, forming raw data into network structure can be beneficial in data storage and has lower computation complexity in particular cases comparing to other relational structure \cite{6}.

\bigskip


\begin{figure}
\centering
\includegraphics[width=0.5\textwidth]{fig_1.png}
\caption{\label{fig:fig_1}A network with unique ID and color labels. In this network, there are two communities whose nodes are either colored in sky blue or in grass green.}
\end{figure}


\subsection{Community}

In network analysis, community in a network often refers to a group of vertex that has closer relation comparing to other vertices that are not in the same group. To simplify the complexity of the context, in this dissertation, no vertex will be assigned to multiple communities, that is $\forall C_i,\ C_j \text{, } i \neq j$, $C_i \cap C_j = \phi$. Depending on the content of network, such as social network, chemical network, and ecosystem network, the definition of forming a community may vary. Furthermore, if given a weight network such as a geographical network, we can intuitively determine if vertices can be grouped into a community by taking their distance(weight of edges) into consideration. Directed network can also be partitioned into communities if the community detection is designed carefully with the help of linear algebra and information theory\cite{4}. Without doubt, there are networks that has no intention of creating communities such as random graphs or Barabási–Albert model\cite{5}; researchers can still apply community detection algorithms to unveil the latent structure from these networks.

\section{Goodness of Partition}

Intuitively, to examine if a partition is 'good' or not, researchers should measure following two properties: 
\begin{enumerate}[label=(\alph*)]
\item Tightness: Vertices within the same community should have strong relation.
\item Sparsity: Communities should have weak relation mutually.
\end{enumerate}

In this section, a number of approaches will be introduced to evaluate the partition considering the properties mentioned above. Other additional properties such as betweeness, conductivity, and modularity are often take into consideration for better understanding of specific types of real-world network\cite{7,8}; some additional properties will also be explained in this section.

\subsection{Performance Evaluation}
A good partition should have high tightness intra-community-wise and strong trait of sparsity inter-community-wise. Given a undirected partitioned network, a performance function $P(G)$ is defined as
$$P(G) = \frac{\text{edge of intra-community} + \text{non-edge of inter-community}}{\text{total potential edges}}$$
$$=\frac{|E_{intra}| + [n*(n-1) - |E_{intra}| - |E_{inter}|]}{n*(n-1)/2}$$
$$=2*\left[ 1 - \frac{ |E_{inter}| } { n*\left( n-1 \right) }  \right],$$
where G is a partitioned network, $E_{intra}$ is the set of edges lie within communities, $E_{inter}$ is the set of edges connecting vertices between two different communities. After a series of simplification, it is not difficult to observe that a network has smaller set of inter-community edges can leads to higher value of $P(G)$(better performance)\cite{7}. In the aspect of algorithm complexity analysis, the time complexity of performance evaluation is in $O(E)$, where E is the edge set; with linear time complexity, researchers can conduct faster initial evaluation comparing to other advanced evaluation procedures.

\subsection{Mean Square Error}
For networks with coordinates provided, take geographical network or cosmic network for instance \cite{9,10}, computing the mean square error(MSE) evaluation can give us an idea if a partition is good. A demonstration of geographical network is shown in figure \ref{fig:fig_2}. The quality function of MSE evaluation is defined as follows:
    $$Q(G) = \sum_{i}^{k}\sum_{j}^{n} M_{i j} * |v_{j} - \bar{v}_{i}|^{2},$$
where G is a partitioned network, k is the total number of communities in a partition, n is the count of the vertices of network $G$, $M$ is a k by n communities matrix, \[
  M_{ij} = 
  \begin{cases}
    1, &  v_{j} \in C_i\\
    0, & \text{otherwise}
  \end{cases}
\]
, and $\bar{v}_{i}$ is the arithmetic center of $C_i$. 

\bigbreak

This evaluation is relatively straightforward; if given a well-partitioned network, it is expected vertices within the same community cluster densely and surrounds the center of the community; such behavior results in mean square error within every community approaches to zero. The computation complexity of MSE evaluation is in $O(C*V)$, the computing time can be reduced by pre-allocating vertices into a collection of set distributed by the partition.

\begin{figure}
\centering
\includegraphics[width=0.6\textwidth]{fig_2.png}

\caption{\label{fig:fig_2}An airline flight network of the United States of America\cite{11}.}
\end{figure}

\subsection{Modularity}

This evaluation method is first introduced by Blondel in 2008 \cite{12}. When tackling with a humongous and complex network, a fast and approximated evaluation algorithm is necessary for time and resource efficiency. In this end, the modularity evaluation stands out to do this job. The modularity function is defined as follows:
$$ Q = \frac{1}{2m} \sum_{i,j} \left[ A_{i j} - \frac{k_{i}k_{j}}{2m}\right] \delta(c_i, c_j),$$
where $m$ is the count of edges, $A$ is the adjacency matrix of network G, $k_i$ is the degree of vertex i if network is both undirected and unweighted, and $\delta$-function takes two labels, $c_i \text{ and } c_j$ as input, returns 1 if $c_i = c_j$ and 0 otherwise.

\bigbreak

Given an undirected and unweighted network, the value of modularity lies in range $[-0.5, 1]$. If the edges within communities exceed the expected count of edges, the $Q$ value is then positive; this implies that the edges inside communities are averagely concentrated. As author stated in the original paper \cite{12}, implementing modularity to examine the partition is not the only usage. The modularity function is as well suitable for partition optimization for large networks. The community detection algorithms that import modularity optimization, such as Louvain method, will be discussed in the latter section.

\bigbreak

Despite the time complexity of computing modularity is in $O(n*log n)$, which is faster comparing to other meta partition evaluations \cite{2, 12}, the modularity evaluation is not capable of detecting small communities. In fact, the evaluation may suggest small communities should be merged into a greater community, causing false estimation of structure and count of communities. Such over-merging behavior is called resolution limit \cite{13}.

%% Read more about modularity and it's concept
%% Read resolution limit (over-merging)

\subsection{Normalized Mutual Information}

%% Read page 3 "Evaluation of Community Detection Methods"

\pagebreak
\begin{thebibliography}{100}

\bibitem{1}
  
    \textit{Community detection algorithms: A comparative analysis},
    Lancichinetti, A. and Fortunato, S.,2009.
    1986.
  
\bibitem{2}
    \textit{Parallel Protein Community Detection in
    Large-scale PPI Networks Based on
    Multi-source Learning}
    J. Chen, K. Li, K. Bilal, A. A. Metwally, K. Li, and P. S. Yu,
    2018
    
\bibitem{3}
    \textit{Community detection in Social Media. Data Mining and Knowledge Discovery}
    Papadopoulos, S., Kompatsiaris, Y., Vakali, A. and Spyridonos, P.,
    2011
    
\bibitem{4}
    \textit{Quantitative methods for ecological network analysis. Computational Biology and Chemistry}
    Ulanowicz, R.E.,
    2004
    
\bibitem{5}
    \textit{tatistical mechanics of complex networks. Reviews of Modern Physics}
    Albert, R. and Barabási, A.-L.,
    2002
    
\bibitem{6}
    \textit{Comparison of Graph Databases and Relational Databases When Handling Large-Scale Social Data}
    Chen, Yaowen,
    2016
    
\bibitem{7}
    \textit{Community detection in graphs}
    Fortunato, S.,
    2010
    
\bibitem{8}
    \textit{Community structure in social and biological networks}
    Girvan, M. and Newman, M.E.J.,
    2002
    
\bibitem{9}
    \textit{Network analysis of the COSMOS galaxy field}
    de Regt, R., Apunevych, S., von Ferber, C., Holovatch, Y. and Novosyadlyj, B,
    2018
    
\bibitem{10}
    \textit{Evaluation of Community Detection Methods}
    Liu, X., Cheng, H.-M. and Zhang, Z.-Y,
    2019
    
\bibitem{11}
    \textit{An Introduction of Gephi}
    2017

\bibitem{12}
    \textit{Fast unfolding of communities in large networks}
    Blondel, V.D., Guillaume, J.-L., Lambiotte, R. and Lefebvre, E
    2008
    
\bibitem{13}
    \textit{Resolution limit in community detection}
    Fortunato, S. and Barthelemy, M.,
    2006



        
\end{thebibliography}

\end{document}